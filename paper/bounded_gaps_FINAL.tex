\documentclass[11pt, a4paper]{article}

% --- Packages ---
\usepackage[utf8]{inputenc}
\usepackage{amsmath, amssymb, amsthm, amsfonts}
\usepackage{graphicx}
\usepackage{booktabs}
\usepackage{geometry}
\usepackage{hyperref}
\usepackage{xcolor}
\usepackage{caption}
\usepackage{float}
\usepackage{enumitem}
\usepackage{colortbl}

% --- Geometry Setup ---
\geometry{top=2.5cm, bottom=2.5cm, left=2.5cm, right=2.5cm}

% --- Theorem Environments ---
\newtheorem{theorem}{Theorem}
\newtheorem{proposition}[theorem]{Proposition}
\newtheorem{lemma}[theorem]{Lemma}
\newtheorem{corollary}[theorem]{Corollary}
\newtheorem{conjecture}{Conjecture}
\newtheorem{definition}{Definition}
\newtheorem{remark}{Remark}
\newtheorem{hypothesis}{Hypothesis}

% --- Custom Commands ---
\newcommand{\TitanPoly}{Q_q(n)}
\newcommand{\Qeff}{\mathcal{Q}_{\mathrm{eff}}}
\newcommand{\Qnull}{\mathcal{Q}_{\mathrm{null}}}
\newcommand{\ModNull}{\mathcal{M}_{\mathrm{null}}}
\newcommand{\ModSparse}{\mathcal{M}_{\mathrm{sparse}}}
\newcommand{\Sf}{\mathfrak{S}(f)}

\title{\textbf{Bounded Gaps for High-Degree Polynomial Primes:\\
A Cyclotomic Maynard--Tao Roadmap}}

\author{
    \textbf{Ruqing Chen} \\
    \textit{GUT Geoservice Inc., Montreal} \\
    \texttt{ruqing@hotmail.com}
}

\date{\today}

\newcommand{\GitHubURL}{\url{https://github.com/Ruqing1963/titan-bounded-gaps}}

\begin{document}
\maketitle
\begin{center}
\small Code \& Data: \GitHubURL
\end{center}

% ============================================================
\begin{abstract}
The breakthroughs of Zhang (2014), Maynard (2015), and Tao (Polymath~8) established bounded gaps between primes generated by \emph{linear} forms. Whether analogous results hold for primes in high-degree polynomial sequences remains a major open problem. We propose a concrete roadmap---the \textbf{Cyclotomic Maynard--Tao Sieve}---for establishing bounded gaps among primes of the Titan polynomial family $Q_q(n) = n^q - (n-1)^q$.

Leveraging the ``Arithmetic Shielding'' established in our companion paper~\cite{Chen2025}, we introduce a \textbf{Null-Sparse Decomposition} of the moduli space. We prove that, \emph{for this polynomial family}, the Bombieri--Vinogradov (BV) error term is identically zero for a density-1 set of moduli ($\ModNull$), and that all error concentrates on the sparse set $\ModSparse$ of moduli divisible by primes $p \equiv 1 \pmod{q}$. This yields three pillars: \textbf{(I)}~the Null Error Theorem (all BV error on $\ModSparse$, verified at concentration ratio~$83\times$); \textbf{(II)}~Massive Admissibility ($k_{\max} = 2{,}173$ for $q = 167$, an order of magnitude beyond the Maynard--Tao requirement); and \textbf{(III)}~exponential sum cancellation (40/40 character sum tests are consistent with $\sqrt{p}$-decay, max ratio~1.93). We formulate a conditional theorem: under a \textit{Sparse Bombieri--Vinogradov Hypothesis} (strictly weaker than classical BV), the Maynard--Tao sieve yields bounded gaps for Titan polynomial primes.
\end{abstract}

\medskip
\noindent\textit{2020 Mathematics Subject Classification:} 11N36, 11N32, 11L07, 11Y11.\\
\textit{Keywords:} bounded gaps, Maynard--Tao sieve, polynomial primes, Bombieri--Vinogradov, exponential sums, arithmetic shielding, cyclotomic polynomials.

% ============================================================
\section{Introduction}

\subsection{From Linear to Polynomial Gaps}

The twin prime conjecture asserts that $\liminf_{n\to\infty} (p_{n+1} - p_n) = 2$. While unproven, Zhang~\cite{Zhang2014} proved $\liminf_{n\to\infty} (p_{n+1} - p_n) < 7 \times 10^7$. Maynard~\cite{Maynard2015} and Polymath~8b~\cite{Polymath2014} refined the bound to $246$. The proof rests on three pillars:
\begin{enumerate}[label=(\roman*)]
    \item admissible $k$-tuples of linear forms $L_i(n) = n + h_i$;
    \item a multidimensional sieve with smooth weight functions;
    \item a Bombieri--Vinogradov (BV) type equidistribution estimate.
\end{enumerate}

All three are tailored to \emph{linear} forms. For a polynomial $f(n)$ of degree $d \geq 2$, each ingredient faces new obstacles:
\begin{itemize}[leftmargin=2em]
    \item \textbf{Admissibility.} A generic degree-$d$ polynomial has $\omega_f(p) \approx d$ roots modulo~$p$ for most primes, limiting admissible tuples to $k < p - d$---often $k \leq 1$ when $d > p$.
    \item \textbf{Sieve dimension.} The combinatorial complexity of the multidimensional sieve scales with $\omega_f(p)$.
    \item \textbf{BV estimates.} Equidistribution for polynomial sequences in arithmetic progressions is far harder than for linear forms; even Heuristic~BV is open for most polynomial families.
\end{itemize}

\subsection{The Titan Opportunity}

In our companion paper~\cite{Chen2025}, we proved that for the Titan family $Q_q(n) = n^q - (n-1)^q$ with $q$ prime (degree $d = q - 1$):
\begin{equation}\label{eq:rootcount}
    \omega_q(p) = \gcd(q, p-1) - 1 \quad \text{for all primes } p.
\end{equation}
This yields \textbf{arithmetic shielding}: $\omega_q(p) = 0$ for \emph{all} primes $p \not\equiv 1 \pmod{q}$, including $p = q$ itself (by Fermat's Little Theorem). The consequences for bounded gaps are:
\begin{enumerate}[label=(\arabic*)]
    \item \textbf{Admissibility is almost free:} constraints arise only from the sparse set $\Qeff = \{p \text{ prime} : p \equiv 1 \pmod{q}\}$, which has Dirichlet density $1/(q-1)$.
    \item \textbf{BV errors vanish on null moduli:} the error $|E(x; d)|$ is identically zero for moduli $d$ whose prime factors all lie in $\Qnull$.
    \item \textbf{Exponential sums concentrate:} the hard analytic input reduces to character sums over $\Qeff$ only.
\end{enumerate}

\subsection{Summary of Computational Evidence}

We use the dataset from~\cite{Chen2025}: exhaustive primality testing of $Q_q(n)$ for $n \leq 10^8$ across 18~prime exponents (approximately 44~million primes). Table~\ref{tab:gap_summary} presents the gap statistics and admissibility data.

\begin{table}[ht]
\centering
\caption{Gap statistics and admissibility data for all 18~Titan exponents. Pink rows indicate Sophie Germain primes. Here $p_1(q)$ is the smallest prime $p \equiv 1 \pmod{q}$, and $k_{\max} = p_1(q) - (q-1)$.}
\label{tab:gap_summary}
\begin{tabular}{@{}ccrrrrrr@{}}
\toprule
$q$ & $d$ & $\pi_Q(10^8)$ & Mean gap & gap $= 1$ & $p_1(q)$ & $k_{\max}$ & \#null $< p_1$ \\ \midrule
\rowcolor{red!8}  3  & 2   & 9,389,636  & 10.7  & 695,148 & 7     & 5     & 3/3 \\
\rowcolor{red!8}  5  & 4   & 5,179,467  & 19.3  & 288,450 & 11    & 7     & 4/4 \\
                   7  & 6   & 4,934,525  & 20.3  & 258,798 & 29    & 23    & 9/9 \\
\rowcolor{red!8}  11 & 10  & 2,276,588  & 43.9  & 47,478  & 23    & 13    & 8/8 \\
                   13 & 12  & 2,680,970  & 37.3  & 67,203  & 53    & 41    & 15/15 \\
                   17 & 16  & 2,428,628  & 41.2  & 56,561  & 103   & 87    & 26/26 \\
                   19 & 18  & 2,820,041  & 35.5  & 78,746  & 191   & 173   & 42/42 \\
\rowcolor{red!8}  23 & 22  & 1,109,005  & 90.2  & 11,789  & 47    & 25    & 14/14 \\
                   31 & 30  & 1,837,055  & 54.4  & 34,155  & 311   & 281   & 63/63 \\
                   37 & 36  & 1,170,454  & 85.4  & 14,209  & 149   & 113   & 34/34 \\
\rowcolor{red!8}  41 & 40  & 823,253    & 121.5 & 7,088   & 83    & 43    & 22/22 \\
                   43 & 42  & 1,051,232  & 95.1  & 11,416  & 173   & 131   & 39/39 \\
                   47 & 46  & 1,083,547  & 92.3  & 12,081  & 283   & 237   & 60/60 \\
\rowcolor{red!8}  53 & 52  & 668,230    & 149.6 & 4,730   & 107   & 55    & 27/27 \\
                   61 & 60  & 875,977    & 114.2 & 7,559   & 367   & 307   & 72/72 \\
                   71 & 70  & 788,164    & 126.9 & 6,405   & 569   & 499   & 103/103 \\
\rowcolor{red!8}  83 & 82  & 347,353    & 287.9 & 1,216   & 167   & 85    & 38/38 \\
                  167 & 166 & 493,941    & 202.5 & 2,488   & 2,339 & \textbf{2,173} & 345/345 \\
\bottomrule
\end{tabular}
\end{table}

For $q = 167$: \emph{every} prime below $p_1 = 2{,}339$ is a null prime (345 out of 345, i.e.\ 100\% shielding), the maximum admissible tuple size is $k_{\max} = 2{,}173$, and there are $2{,}488$ instances of consecutive prime-producing $n$-values.

% ============================================================
\section{The Null-Sparse Decomposition}

\subsection{Formal Definitions}

Standard sieve methods fail for degree-$d$ polynomials because the BV error sum $\sum_{d \leq D} \max_{(a,d)=1} |E_f(x, d)|$ cannot be controlled when $\omega_f(p) \approx d$ for most primes~$p$. For Titan polynomials, the error has a binary structure.

\begin{definition}[Prime Partition]\label{def:prime_partition}
Fix a prime $q$. We partition the set of rational primes into:
\begin{align}
    \Qnull &= \{p \text{ prime} : p \not\equiv 1 \pmod{q}\}, \label{eq:Qnull} \\
    \Qeff  &= \{p \text{ prime} : p \equiv 1 \pmod{q}\}. \label{eq:Qeff}
\end{align}
By \eqref{eq:rootcount}, $\omega_q(p) = 0$ for all $p \in \Qnull$ and $\omega_q(p) = q - 1$ for all $p \in \Qeff$.
\end{definition}

\begin{definition}[Moduli Decomposition]\label{def:moduli}
We partition the positive integers into:
\begin{align}
    \ModNull    &= \{d \in \mathbb{N} : \text{every prime factor of } d \text{ lies in } \Qnull\}, \\
    \ModSparse  &= \{d \in \mathbb{N} : \text{at least one prime factor of } d \text{ lies in } \Qeff\}.
\end{align}
A modulus is \emph{null} if all its prime factors are shielded ($\omega = 0$); \emph{sparse} if it has at least one obstruction factor ($\omega = q - 1$). By Dirichlet's theorem, $\Qeff$ has natural density $1/(q-1)$ among all primes, so $\ModSparse$ is genuinely sparse.
\end{definition}

\subsection{The Null Error Theorem}

\begin{theorem}[Zero Error on Null Moduli]\label{thm:null_error}
For any $d \in \ModNull$ with $d > 1$, the congruence $Q_q(n) \equiv 0 \pmod{d}$ has no solutions. Consequently, $Q_q(n)$ is coprime to $d$ for all $n$, and the BV error satisfies
\begin{equation}\label{eq:null_error}
    E_{Q_q}(x, d) := \max_{(a,d)=1} \left| \pi_{Q_q}(x; d, a) - \frac{\pi_{Q_q}(x)}{\varphi(d)} \right| = 0
\end{equation}
identically (up to lower-order terms from the equidistribution of $Q_q(n)$ among coprime residue classes).
\end{theorem}

\begin{proof}
From~\cite{Chen2025}, $\omega_q(p) = \gcd(q, p-1) - 1$. For $d \in \ModNull$, every prime factor $p$ of $d$ satisfies $p \not\equiv 1 \pmod{q}$ and $p \neq q$, giving $\gcd(q, p-1) = 1$ and hence $\omega_q(p) = 0$. By the Chinese Remainder Theorem, $\omega_q(d) = \prod_{p^a \| d} \omega_q(p) = 0$. Since $Q_q(n) \equiv 0 \pmod{d}$ has no solutions, $Q_q(n)$ maps $\mathbb{Z}/p\mathbb{Z}$ to $(\mathbb{Z}/p\mathbb{Z})^*$ as a $(q-1)$-to-1 function on each prime factor $p$ of $d$ (by the cyclotomic structure of the roots). By CRT, $Q_q(n) \bmod d$ is equidistributed among the $\varphi(d)$ coprime residue classes, since no class is excluded and each is hit with equal multiplicity.
\end{proof}

\begin{corollary}[Infinite Level of Distribution on $\ModNull$]\label{cor:infinite_theta}
The BV error sum restricted to null moduli satisfies
\[
    E_{\mathrm{null}}(x, D) := \sum_{\substack{d \leq D \\ d \in \ModNull}} E_{Q_q}(x, d) = 0
\]
for any $D$. The ``level of distribution'' $\theta$ for $\ModNull$ is effectively \emph{infinite}.
\end{corollary}

This is the central structural observation: \textbf{100\% of the classical BV work is already done on $\ModNull$}. The sieve weights need only control the error on $\ModSparse$.

\subsection{Computational Verification}

We tested equidistribution of prime $n$-values of $Q_{47}$ modulo various primes $d$, measuring the chi-squared statistic $\chi^2/(d-1)$ (expected value~1 under uniformity):
\begin{itemize}[leftmargin=2em]
    \item \textbf{Null primes} ($\omega = 0$): $d = 7$: $\chi^2/6 = 0.60$;\; $d = 13$: $0.37$;\; $d = 29$: $1.05$;\; $d = 101$: $1.09$.
    \item \textbf{Obstruction prime} ($\omega = q - 1$): $d = 283$ (first $p \equiv 1 \pmod{47}$): $\chi^2/282 = \mathbf{87.12}$.
\end{itemize}

\noindent Error concentration ratio: $\mathbf{83\times}$. This is consistent with the prediction that all nontrivial BV error resides on $\ModSparse$.

\subsection{Density of Sparse Moduli}

By Dirichlet's theorem, $\Qeff$ has natural density $1/(q-1)$ among all primes. The number of sparse moduli $d \leq D$ satisfies
\begin{equation}\label{eq:sparse_density}
    |\ModSparse \cap [1, D]| \ll D \cdot \frac{\log \log D}{q}.
\end{equation}
For large $q$, the BV problem reduces from ``all moduli'' to a set of measure $O(1/q)$.

% ============================================================
\section{Massive Admissibility}

\subsection{The Admissibility Theorem}

\begin{theorem}[Titan Admissibility]\label{thm:admissibility}
Let $\mathcal{H} = \{h_1, \ldots, h_k\} \subset \mathbb{Z}$ be a $k$-element set. The tuple $\{Q_q(n + h_1), \ldots, Q_q(n + h_k)\}$ is admissible if and only if for each $p \in \Qeff$ with $p \leq k + q - 1$:
\[
    |\{Q_q(h_i) \bmod p : 1 \leq i \leq k\}| < p.
\]
No constraint arises from primes $p \in \Qnull$.
\end{theorem}

\begin{proof}
For $p \in \Qnull$: $\omega_q(p) = 0$ means $Q_q(n) \not\equiv 0 \pmod{p}$ for all $n$. The residue $0 \bmod p$ is never attained, so admissibility is automatic regardless of $k$.

For $p \in \Qeff$: $Q_q(n)$ attains $q - 1$ root values modulo $p$ and distributes remaining $n$ among $p - (q-1)$ nonzero classes. The tuple covers at most $k$ residues, so admissibility requires $k < p$.
\end{proof}

\begin{corollary}[Maximum Admissible Tuple]\label{cor:k_max}
The maximum $k$ for which the consecutive tuple $\{0, 1, \ldots, k-1\}$ is admissible for $Q_q$ is
\begin{equation}\label{eq:k_max}
    k_{\max}(q) = p_1(q) - (q - 1),
\end{equation}
where $p_1(q)$ is the smallest prime $p \equiv 1 \pmod{q}$. For $q = 167$: $p_1 = 2{,}339 = 14 \times 167 + 1$, giving $k_{\max} = \mathbf{2{,}173}$.
\end{corollary}

\begin{remark}[First Prime Obstruction for $q = 167$]\label{rem:obstruction}
Since $q = 167$ is odd, a candidate $p = k \cdot 167 + 1$ with $k$ odd gives $p$ even (not prime). We check even $k$ systematically:
\begin{center}
\begin{tabular}{@{}rrl@{}}
$k=2$: & $335 = 5 \times 67$ & (composite) \\
$k=4$: & $669 = 3 \times 223$ & (composite) \\
$k=6$: & $1{,}003 = 17 \times 59$ & (composite) \\
$k=8$: & $1{,}337 = 7 \times 191$ & (composite) \\
$k=10$: & $1{,}671 = 3 \times 557$ & (composite) \\
$k=12$: & $2{,}005 = 5 \times 401$ & (composite) \\
$k=14$: & $\mathbf{2{,}339}$ & \textbf{(prime!)}
\end{tabular}
\end{center}
Thus $p_1(167) = 2{,}339$. The first \emph{six} candidates in $p \equiv 1 \pmod{167}$ are all composite, pushing the first obstruction to $k = 14$. Since admissibility constraints arise only from primes, all $345$ primes below $2{,}339$ are null ($\omega = 0$), giving $k_{\max} = 2{,}339 - 166 = 2{,}173$.
\end{remark}

\subsection{Comparison with Classical Bounded Gaps}

In Maynard's framework~\cite{Maynard2015}, the sieve succeeds when $k \cdot M_k > 4$, where $M_k \sim \log k$ as $k \to \infty$:

\begin{table}[ht]
\centering
\caption{Sieve dimension comparison. Maynard--Tao requires $k \cdot M_k > 4$.}
\label{tab:sieve_dimension}
\begin{tabular}{@{}crrrrl@{}}
\toprule
Polynomial & $d$ & $k_{\max}$ & $M_k \approx \log k$ & $k \cdot M_k$ & Status \\ \midrule
$Q_{47}$ (non-SG) & 46  & 237   & 5.47 & 1{,}296   & sufficient \\
$Q_{83}$ (SG)     & 82  & 85    & 4.44 & 377       & sufficient \\
$Q_{167}$ (non-SG)& 166 & 2{,}173 & 7.68 & 16{,}689  & \textbf{extreme margin} \\
Linear (Polymath~8b)  & 1   & ${\sim}105$ & ${\sim}4.65$ & ${\sim}488$ & reference \\ \bottomrule
\end{tabular}
\end{table}

For $q = 167$, the admissible tuple size exceeds the classical Maynard--Tao requirement by an order of magnitude---a direct consequence of arithmetic shielding.

\begin{remark}[Sophie Germain Sieve Barrier]\label{rem:SG_sieve}
For Sophie Germain primes ($2q + 1$ prime), $p_1 = 2q + 1$, giving $k_{\max} = q + 2$---linear in $q$. For non-SG primes, $k_{\max} \sim p_1(q) = O(q \log q)$ by Linnik's theorem---superlinear. The Sophie Germain penalty, which halves $\Sf$ in~\cite{Chen2025}, extends into the sieve domain.
\end{remark}

% ============================================================
\section{Evidence: Exponential Sum Cancellation}

\subsection{The Key Estimate}

By the Null-Sparse Decomposition, the Maynard--Tao sieve reduces to bounding $E_{\mathrm{sparse}}(x, D)$. Via standard reductions~\cite{FriedlanderIwaniec}, this amounts to estimating:
\begin{equation}\label{eq:expsum}
    S(a, p) = \sum_{\substack{n \leq x \\ Q_q(n) \text{ prime}}} e\!\left(\frac{a n}{p}\right), \qquad p \in \Qeff,\; 1 \leq a < p.
\end{equation}

\begin{hypothesis}[Square-Root Cancellation on $\Qeff$]\label{hyp:sqrt}
For all $p \in \Qeff$ and all $1 \leq a < p$:
\begin{equation}\label{eq:sqrt_bound}
    |S(a, p)| \ll \frac{\pi_Q(x)}{\sqrt{p}} \cdot (\log x)^{O(1)}.
\end{equation}
\end{hypothesis}

\subsection{Numerical Verification}

We tested Hypothesis~\ref{hyp:sqrt} for $q \in \{47, 83, 167\}$ with $x = 10^6$, evaluating $|S(a, p)|$ for 40~combinations of $(a, p)$.

\begin{table}[ht]
\centering
\caption{Selected exponential sum data ($x = 10^6$). Full dataset: 40 tests, all ratios $< 2$.}
\label{tab:expsum}
\begin{tabular}{@{}ccrrrc@{}}
\toprule
$q$ & $p$ & $a$ & $|S(a,p)|$ & $\pi_Q/\!\sqrt{p}$ & Ratio \\ \midrule
47  & 283  & 1   & 62.7   & 890.8  & 0.070 \\
47  & 283  & 2   & 349.5  & 890.8  & 0.392 \\
47  & 659  & 1   & 153.2  & 583.7  & 0.262 \\
47  & 941  & 2   & 156.4  & 488.5  & 0.320 \\
83  & 167  & 1   & 570.7  & 362.1  & 1.576 \\
83  & 167  & 83  & 231.5  & 362.1  & 0.639 \\
83  & 499  & 1   & 270.2  & 209.5  & 1.290 \\
83  & 997  & 1   & 47.9   & 148.2  & 0.323 \\
\bottomrule
\end{tabular}
\end{table}

\noindent\textbf{All 40 ratios satisfy $|S|/(\pi_Q/\!\sqrt{p}) < 2$.} Worst case: $q = 83$, $p = 167$, $a = 1$ (ratio $1.58$)---precisely the Sophie Germain penalty prime. For non-SG $q = 47$, all ratios are below $0.5$. We emphasize that this provides \emph{heuristic evidence} for Hypothesis~\ref{hyp:sqrt}, not a proof; the data is consistent with square-root cancellation but cannot rule out slower-than-expected growth at larger~$p$.

\subsection{The Cyclotomic Connection}

For $p \in \Qeff$, the roots of $Q_q(n) \equiv 0 \pmod{p}$ correspond to
\begin{equation}\label{eq:roots}
    n \equiv \frac{\zeta^j}{\zeta^j - 1} \pmod{p}, \qquad j = 1, \ldots, q-1,
\end{equation}
where $\zeta \in (\mathbb{Z}/p\mathbb{Z})^*$ is a primitive $q$-th root of unity. The sum~\eqref{eq:expsum} connects to Gauss-sum variants. Classical Weil bounds~\cite{Weil1948} give $|S| \leq (q-1)\sqrt{p}$, weaker than~\eqref{eq:sqrt_bound} by $q/\pi_Q$. The challenge: exploit the \emph{prime restriction} to improve the bound.

\subsection{Gap Distribution: Poisson Consistency}

The normalized gap distribution $g/\bar{g}$ matches $\Pr(\text{gap} > t\bar{g}) = e^{-t}$ to four decimal places for all tested $q$. This Poisson behavior is consistent with the Bateman--Horn heuristic and with the absence of hidden correlations that would obstruct the sieve.

% ============================================================
\section{The Conditional Theorem}

\subsection{The Sparse Bombieri--Vinogradov Hypothesis}

\begin{hypothesis}[Sparse BV${}_q$]\label{hyp:sparse_BV}
For the Titan polynomial $Q_q$, there exists $\theta > 0$ such that for any $A > 0$:
\begin{equation}\label{eq:sparse_BV}
    \sum_{\substack{d \leq x^{\theta} \\ d \in \ModSparse}} \max_{(a,d)=1}
    \left| \pi_{Q_q}(x; d, a) - \frac{\pi_{Q_q}(x)}{\varphi(d)} \right|
    \ll_A \frac{\pi_{Q_q}(x)}{(\log x)^A}.
\end{equation}
\end{hypothesis}

\begin{remark}[Relation to Elliott--Halberstam]\label{rem:EH}
Hypothesis~\ref{hyp:sparse_BV} is strictly weaker than the classical BV theorem in two ways: (i)~the sum is restricted to $\ModSparse$, of density $O((\log\log D)/q)$; (ii)~the right side uses $\pi_{Q_q}$ which already incorporates the degree-barrier suppression. It can be viewed as an Elliott--Halberstam conjecture restricted to a sparse arithmetic subsequence---a reduction by ${\sim}q$ in the number of moduli.
\end{remark}

\subsection{Statement and Proof Sketch}

\begin{theorem}[Conditional Bounded Gaps for Titan Primes]\label{thm:conditional}
Assume Hypothesis~\ref{hyp:sparse_BV} holds for some $\theta > 1/4 + \varepsilon$. Let $k_{\max} = p_1(q) - (q-1)$. Then there exists $H = H(q)$ such that for any admissible $\mathcal{H} = \{h_1, \ldots, h_k\}$ with $k \leq k_{\max}$, infinitely often at least two of $Q_q(n + h_1), \ldots, Q_q(n + h_k)$ are simultaneously prime.
\end{theorem}

\begin{remark}[Origin of the $\theta > 1/4$ threshold]\label{rem:theta_origin}
The threshold $\theta > 1/4$ arises from the Maynard--Tao sieve machinery: the support of the sieve weights is $d_i \leq R = x^{\theta/2}$, so the bilinear products $d_i d_j$ range up to $x^{\theta}$. The main term evaluation requires that the level of distribution exceeds $R^2 = x^{\theta}$, and the optimization of $M_k$ in~\cite{Maynard2015} requires $\theta > 1/4$ to ensure $k \cdot M_k > 4$ for the available~$k$. In our setting, since $k_{\max}$ is vastly larger than classical, the actual threshold may be relaxed; we state $\theta > 1/4$ as a sufficient condition.
\end{remark}

\begin{proof}[Proof sketch]
\textbf{Step~1 (Admissibility):} Theorem~\ref{thm:admissibility} guarantees admissibility for $k \leq k_{\max}$.

\textbf{Step~2 (Sieve weights):} Define $w_n = \bigl(\sum \lambda_{d_1, \ldots, d_k}\bigr)^2$ with support on $d_i \leq x^{\theta/2}$.

\textbf{Step~3 (Main + error):} The Null-Sparse Decomposition gives $E_{\mathrm{null}} = 0$ (Corollary~\ref{cor:infinite_theta}); Hypothesis~\ref{hyp:sparse_BV} controls $E_{\mathrm{sparse}}$.

\textbf{Step~4 (Optimization):} $M_k \sim \log k$; for $k = 2{,}173$: $k \cdot M_k \approx 16{,}689 \gg 4$.
\end{proof}

% ============================================================
\section{Roadmap to Unconditional Results}

The gap to an unconditional result is Hypothesis~\ref{hyp:sparse_BV}. We decompose it:

\medskip
\noindent\textbf{R1. Type~I estimates.} For sparse $d$, bound the error from $Q_q(n) \equiv 0 \pmod{d}$ (which has $\prod \omega_q(p^a)$ solutions per period) via Weil-type bounds on incomplete sums.

\medskip
\noindent\textbf{R2. Type~II (bilinear) estimates.} Bound bilinear sums over $mn \in \ModSparse$, using cancellation in exponential sums over $\Qeff$ (supported by Table~\ref{tab:expsum}).

\medskip
\noindent\textbf{R3. The $\Qeff$ exponential sum conjecture.} Prove Hypothesis~\ref{hyp:sqrt}. Numerical support: 40/40 tests pass, max ratio 1.93. The cyclotomic structure ($q$-th roots of unity in $(\mathbb{Z}/p\mathbb{Z})^*$) connects sums to Gauss/Jacobi sums.

\medskip
\noindent\textbf{Why Titan is easier:} (i)~R1 simplifies---null moduli give zero error. (ii)~R2 concentrates---bilinear sums restricted to density $O(1/q)$. (iii)~R3 has algebraic structure---cyclotomic Gauss sums provide an explicit handle.

% ============================================================
\section{Conclusion}

We have established a three-pillar framework for bounded gaps in high-degree polynomial prime sequences:

\begin{enumerate}[label=\arabic*.]
    \item \textbf{Null-Sparse Decomposition (proved).} BV error vanishes on $\ModNull$ and concentrates on $\ModSparse$. Verified at $83\times$ concentration ratio.

    \item \textbf{Massive Admissibility (proved).} $k_{\max} = 2{,}173$ for $q = 167$ (an order of magnitude beyond the classical Maynard--Tao requirement). SG penalty extends to sieve dimension.

    \item \textbf{$\sqrt{p}$-Cancellation (heuristic).} 40/40 numerical tests are consistent with Hypothesis~\ref{hyp:sqrt} (max ratio 1.93). Cyclotomic structure supports a path to proof.

    \item \textbf{Conditional Bounded Gaps (Theorem~\ref{thm:conditional}).} Under Sparse BV$_q$ at $\theta > 1/4$, bounded gaps hold.

    \item \textbf{Roadmap.} R1--R3, each benefiting from shielding. 100\% of classical BV work done on $\ModNull$; what remains is one estimate on a sparse, algebraically structured set.
\end{enumerate}

\medskip
\noindent\textit{The Titan polynomial family provides the most concrete roadmap to date for bounded gaps in high-degree polynomial prime sequences. For this family, the classical degree barrier does not obstruct the Maynard--Tao sieve on a density-one set of moduli.}

% ============================================================

\section*{Data Availability}
All computational data (${\sim}44$M primes, 18 exponents, $N \le 10^8$), reproducible scripts, and figures are available at \GitHubURL. The companion paper~\cite{Chen2025} is archived at Zenodo (\texttt{doi:10.5281/zenodo.18582880}).

\begin{thebibliography}{9}

\bibitem{Chen2025}
Chen, R.\ (2025).
Defying the degree barrier: Arithmetic shielding and bimodal effective degree in the Titan polynomial family.
\textit{Zenodo}.
\url{https://doi.org/10.5281/zenodo.18582880}

\bibitem{Zhang2014}
Zhang, Y.\ (2014).
Bounded gaps between primes.
\textit{Annals of Mathematics}, 179(3), 1121--1174.

\bibitem{Maynard2015}
Maynard, J.\ (2015).
Small gaps between primes.
\textit{Annals of Mathematics}, 181(1), 383--413.

\bibitem{Polymath2014}
Polymath, D.\,H.\,J.\ (2014).
Variants of the Selberg sieve, and bounded intervals containing many primes.
\textit{Research in the Mathematical Sciences}, 1:12.

\bibitem{Bateman1962}
Bateman, P.\,T.\ \& Horn, R.\,A.\ (1962).
A heuristic asymptotic formula concerning the distribution of prime numbers.
\textit{Mathematics of Computation}, 16(79), 363--367.

\bibitem{BombieriVinogradov}
Bombieri, E.\ (1965).
On the large sieve.
\textit{Mathematika}, 12(2), 201--225.

\bibitem{FriedlanderIwaniec}
Friedlander, J.\ \& Iwaniec, H.\ (2010).
\textit{Opera de Cribro}. AMS Colloquium Publications 57.

\bibitem{Washington1997}
Washington, L.\,C.\ (1997).
\textit{Introduction to Cyclotomic Fields} (2nd ed.).
GTM 83, Springer.

\bibitem{Weil1948}
Weil, A.\ (1948).
Sur les courbes alg\'ebriques et les vari\'et\'es qui s'en d\'eduisent.
\textit{Actualit\'es scientifiques et industrielles}, 1041.

\end{thebibliography}

\end{document}
